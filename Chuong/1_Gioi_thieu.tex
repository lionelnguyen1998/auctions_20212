\documentclass[../DoAn.tex]{subfiles}
\begin{document}
\section{Đặt vấn đề}
\label{section:1.1}
Trong thời kỳ phát triển như ngày nay, nhu cầu mua bán, trao đổi hàng hóa của con người ngày càng tăng cao. Để thu hút được khách hàng thì các nhà kinh doanh không ngừng thay đổi cách thức, dịch vụ bán hàng. Nắm bắt được tâm lý của khách hàng, nhiều nhà kinh doanh ngoài việc bán các mặt hàng với giá cố định thì đã tổ chức nhiều phiên đấu giá để kích thích sự cạnh tranh khi mua hàng của người tiêu dùng. Việc này tạo hiệu ứng rất tốt đối với người dùng, họ được tự mình định giá cho sản phẩm của mình, mua được sản phẩm với giá rẻ hơn ngoài thị trường. 

Tuy nhiên khi tổ chức các phiên đấu giá truyền thống thì phát sinh nhiều vấn đề như: thủ tục tổ chức cồng kềnh, hội trường, số lượng người tham gia, thời gian bị giới hạn. Bên cạnh đó sàn đấu giá truyền thống thường chỉ đấu giá những sản phẩm có giá trị lớn như nhà đất, đồ cổ quý hiếm….Vì vậy các sản phẩm có giá trị nhỏ hơn khó tiếp cận với khách hàng. Ngoài ra khi số lượng phiên đấu giá lớn dần thì việc quản lý trở nên rất khó khăn và mất thời gian.

Thêm vào đó thương mại điện tử đang ngày càng phát triển và dần trở thành xu hướng tất yếu của thị trường. Bởi vậy, một website đấu giá trực tuyến ra đời sẽ góp phần giải quyết được các vấn đề nói trên. Khi tổ chức một phiên đấu giá trực tuyến thì người bán không mất chi phí thuê hội trường, số lượng người tham gia, thời gian có thể linh động dễ dàng. Số lượng sản phẩm của phiên đấu giá cũng đa dạng hơn, dễ tiếp cận với số lượng lớn người dùng hơn. Đặc biệt việc quản lý, theo dõi các phiên đấu giá dễ dàng và tiết kiệm thời gian hơn.

Tuy nhiên theo khảo sát của em về thị trường Việt Nam hiện nay, có rất ít website đấu giá trực tuyến và các phiên đấu giá thường hay giới hạn nhóm sản phẩm. Còn các sàn đấu giá nước ngoài thì không phù hợp với người Việt Nam, do vấn đề ngôn ngữ, vận chuyển, đổi trả hàng.

Nhận thấy cả lợi ích và hạn chế của những sản phẩm hiện có. Em quyết định đặt mục tiêu cho đồ án là xây dựng một website đấu giá trực tuyến với mô hình C2C (Customer To Customer - Người dùng tới người dùng).
\section{Mục tiêu và phạm vi đề tài}
\label{section:1.2}
Trên thị trường hiện nay có rất nhiều ứng dụng, website tổ chức các phiên đấu giá trực tuyến ở cả Việt Nam và nước ngoài. Phổ biến nhất phải kể đến là ebay.com  ở thị trường nước ngoài, sohot.vn, lacvietauction.vn của Việt Nam.

Ebay.com thì phổ biến trên nhiều quốc gia, sản phẩm đa dạng, có nhiều thương hiệu nổi tiếng, nhưng hiện nay nó không có bản tiếng Việt, cũng không được ưa chuộng tại Việt Nam vì vấn đề là nơi bán quá xa, quá trình vận chuyển vừa mất thời gian, vừa không đảm bảo được trên quãng đường dài như thế thì các sản phẩm dễ vỡ có còn được nguyên vẹn hay không, rồi vấn đề thất lạc sản phẩm, trả hàng nếu muốn cũng rất khó khăn. Còn ở Việt Nam thì có sohot.vn, khi muốn đấu giá trên sàn này thì cần phải đăng ký qua 5giay.vn, giao diện rất khó sử dụng, bên cạnh đó trên sohot.vn chỉ có một số mặt hàng chủ yếu liên quan đến nội thất gia đình. Ngoài ra có sàn đấu giá lacvietauction.vn cũng rất được ưa chuộng tại Việt Nam, tuy nhiên sàn đấu giá này chủ yếu trưng bày, đấu giá  các sản phẩm có giá trị lớn như căn hộ, đất đai, xe ô tô….việc này cũng gây khó khăn cho nhóm người muốn bán những sản phẩm, đồ dùng có giá trị nhỏ hơn.

Do vậy ở đồ án tốt nghiệp này em muốn xây dựng một website cung cấp cho người dùng Việt Nam trải nghiệm tốt hơn, giao diện dễ sử dụng hơn. Bên cạnh đó để đáp ứng nhiều nhóm người dùng hơn thì phần mềm sẽ không giới hạn hay yêu cầu là chỉ bán một mặt hàng nhất định và không ràng buộc mặt hàng đó phải có giá trị thấp nhất là bao nhiêu. Bên cạnh đó để thuận tiện cho việc liên lạc trao đổi giữa các người dùng với nhau thì phần mềm cũng cung cấp cho người dùng ứng dụng chat realtime để liên lạc trao đổi giữa các bên khi cần. 

Ngoài ra vì muốn hướng tới thêm một nhóm người Nhật Bản tại Việt Nam nên website cung cấp việc chuyển đổi giữa tiếng Việt và tiếng Nhật cũng góp phần nâng cao trải nghiệm người dùng hơn.

\section{Định hướng giải pháp}
\label{section:1.3}
Để xây dựng hệ thống sàn đấu giá theo mục tiêu đề tài như đã trình bày ở trên, em định hướng xây dựng đề tài trên nền tảng web cho cả người dùng và người quản trị hệ thống. 

Công nghệ sử dụng gồm có: ReactJs cho ứng dụng web, PHP framework Laravel cho Server. Mô hình sử dụng: Client - Server, kiến trúc sử dụng MVC (Model - View - Controller). Để xây dựng chức năng chat realtime đồ án sẽ sử dụng thư viện Socket.IO. Socket.IO giúp các bên kết nối với nhau, truyền dữ liệu ngay lập tức thông qua Server trung gian. Socket.IO với đặc trưng là dễ sử dụng nên rất được ưa chuộng bởi các lập trình viên, hỗ trợ  nhiều công nghệ realtime như WebSocket, Flash Socket, AJAX long-polling…

Mô hình sử dụng cho đồ án là Mô hình Client  - Server, là loại mô hình mà máy chủ (Server) sẽ lưu trữ các tài nguyên, quản lý,  xử lý các yêu cầu mà khách hàng tương tác trên màn hình và trả ra kết quả trên màn hình cho khách hàng. Client  - Server là mô hình mà một hay nhiều máy khách (Client) gửi yêu cầu đến máy chủ trung tâm\cite{ClientServer}. Mô hình Client - Server được sử dụng trong nhiều ứng dụng hiện nay như ứng dụng web hay ứng dụng di động. Client-Server chỉ mang đặc điểm của phần mềm mà không hề liên quan đến phần cứng, dễ sử dụng, dễ bảo trì, tái sử dụng, hỗ trợ người dùng nhiều dịch vụ đa dạng và sự tiện dụng bởi khả năng truy cập từ xa. Vì vậy đồ án lựa chọn mô hình này khi xây dựng ứng dụng cho Client, một cho trình duyệt web và một ứng dụng sẽ được cài đặt trên server để phục vụ cho ứng dụng Client thông qua lời gọi RESTful API. 

Bên cạnh đó đồ án lựa chọn kiến trúc MVC (Model-View-Controller). Một kiến trúc khá phổ biến và được sử dụng nhiều trên các ứng dụng web ngày nay. Kiến trúc MVC đơn giản, phân chia ứng dụng thành nhiều lớp dễ sử dụng, dễ triển khai và bảo trì. Phần hiển thị ở Client như là View, còn Server gắn với Model và Controller trong kiến trúc.

Tóm lại, định hướng phát triển đề tài được triển khai như sau: Phía Backend sử dụng Ngôn ngữ PHP framework Laravel, sử dụng thư viện ReactJs để xây dựng view hiển thị trên ứng dụng web, để có được chức năng chat realtime thì sử dụng thư viện Socket.IO, hệ quản trị cơ sở dữ liệu là MySQL.

Về công nghệ sử dụng đồ án sẽ trình bày chi tiết hơn trong Chương 3.
\section{Bố cục đồ án}
\label{section:1.4}
Phần còn lại của báo cáo đồ án tốt nghiệp này được chia làm sáu phần, với nội dung các chương như sau: 

Chương 2 sẽ tập trung vào việc khảo sát hiện trạng thực tế hiện nay, đánh giá ưu nhược điểm của các sản phẩm đã có trên thị trường. Từ đó đưa ra những yêu cầu mà người dùng cần để xây dựng các chức năng cơ bản phù hợp với yêu cầu đó. Trong Chương 2 này đồ án cũng sẽ trình bày chi tiết về quy trình nghiệp vụ, đặc tả những chức năng chính của website. Cuối cùng là phân tích, đánh giá về các yêu cầu chức năng, phi chức năng mà đồ án đã đạt được và chưa đạt được.

Tiếp theo, Chương 3 trình bày về công nghệ sử dụng trong suốt đồ án . Bao gồm các cơ sở lý thuyết cơ bản về công nghệ đó và lý giải tại sao công nghệ được lựa chọn cho đồ án. 

Chương 4 sẽ trình bày chi tiết về quá trình thiết kế kiến trúc, giao diện, lớp, cơ sở dữ liệu; chương này còn nêu lên quá trình xây dựng, kết quả đạt được và quá trình kiểm thử phần mềm cho toàn bộ đồ án này. Việc lựa chọn kiến trúc MVC, hay sử dụng hệ quản trị cơ sở dữ liệu MySQL cũng sẽ được trình bày chi tiết tại chương này. Phần kiểm thử sử dụng kỹ thuật kiểm thử hộp đen các chức năng quan trọng của đồ án. Cuối cùng là kế hoạch triển khai dự án trên thực tế thì có những vấn đề gì cần chú ý, các thông số liên quan.

Những đóng góp của đồ án sẽ được trình bày trong chương 5. Ở chương này sẽ trình bày những đóng góp nổi bật của đồ án như (i) chat realtime giữa các người dùng trên website; (ii) tìm kiếm thông tin sản phẩm theo một số từ khóa nổi bật như tên phiên đấu giá, giá khởi điểm, thời gian bắt đầu, thời gian kết thúc; (iii) website cho phép người dùng đánh giá sản phẩm khi nhận được hàng; (iv) cung cấp hệ thống quản lý website dành cho Admin.

Cuối cùng, chương 6 sẽ là những kết luận và kết quả cuối cùng mà đồ án này đạt được. Phần này sẽ tổng kết, đánh giá những điều đã làm được, chưa làm được, những đóng góp chính của em trong đồ án và các định hướng phát triển trong tương lai để sản phẩm được hoàn thiện hơn.
\end{document}